\documentclass[11pt]{article}

\usepackage[T1]{fontenc}
\usepackage{textcomp}

\usepackage[left=3.5cm, right=3.5cm, top=2cm, bottom=2cm]{geometry}
\usepackage{lmodern}
\usepackage{microtype}
\usepackage[htt]{hyphenat}
\usepackage{enumitem}
\usepackage{graphicx}

\usepackage{xcolor}
\usepackage{titlesec}

\definecolor{accent}{HTML}{4C8047}

\newcommand{\hlsection}[1]{%
  \par\addvspace{10pt}%
  \noindent
  \colorbox{accent}{%
    \ttfamily\bfseries\color{white}%
    \strut#1%
  }%
  \par\addvspace{6pt}%
}

\usepackage{setspace}
\setlength{\parindent}{0pt}
\setlength{\parskip}{1pt}
\setstretch{0.95}

\usepackage{titlesec}
\titlespacing*{\section}{0pt}{1em}{0.5em}

\usepackage{xcolor}
\usepackage[normalem]{ulem}
\usepackage{hyperref}

\definecolor{linkgray}{RGB}{50,50,50}

\newcommand{\myhref}[2]{%
  \href{#1}{\textcolor{linkgray}{\underline{#2}}}%
}

\begin{document}

\section*{\texttt{Engineering Open Source Projects (EOSP)}}

\hspace{1cm}

\texttt{Series of lectures and seminars by Anton Grishin\\
  To be presented to high school students from
  \myhref{https://xn--l1afu.xn--p1ai/}{CPM} in
  \myhref{https://cu.ru}{CU} campus.\\
  The all course materials in the
  \myhref{https://github.com/alchemmist/eosp}{alchemmist/eosp} GitHub
  repository.
}

\hlsection{Abstract}
\textit{This project builds a system to evaluate developer contributions based
  on GitHub activity, implemented as a Python library, a CLI, and a
  Telegram bot. Students practice modular design, testing, CI/CD,
  documentation, and real-world Open Source workflows, focusing on
usability and clean, extensible solutions.}

\hlsection{What is the goal?}
The goal of this project is to build a practical system for assessing
developer or team contributions based on GitHub activity. The system
includes a Python library to calculate contribution metrics, a CLI
for interacting with the library, and a Telegram bot for convenient access.

\hlsection{Who is the teacher?}
\begin{minipage}{0.75\textwidth}
  Anton is a hands-on software developer and Python
  instructor with over four years of experience, contributing to
  dozens of repositories. Anton is a finalist of the Russian
  Championship in Competitive Programming. And winner of
  the \guillemotleft{Science for Life}\guillemotright\ conference.
  Author of blog:
  \myhref{https://alchemmist.xyz}{\footnotesize\texttt{alchemmist.xyz}}.

\end{minipage}
\hspace{0.03\textwidth}
\begin{minipage}{0.2\textwidth}
  \includegraphics[width=6em]{anton.jpg}
\end{minipage}

\hlsection{Why this course?}
HR processes and IT management are increasingly complex, making it
difficult to evaluate developers across productivity, code quality,
skills, and contributions without clear metrics. This course teaches
students real-world Open Source practices while providing hands-on
experience in designing modular, maintainable, and well-documented systems.

\hlsection{What's the methodology?}
The course is practice-driven. From the very first lecture, students
receive a concrete task: to design and build a real project. All
subsequent lectures, materials, and activities are structured to
support this goal, gradually adding the knowledge and tools needed to
complete and refine the project.

\hlsection{Learning Outcomes}
By the end of the course, students will know:
\begin{itemize}[topsep=0pt, partopsep=0pt, itemsep=2pt, parsep=0pt]
  \item How the Open Source development process works in practice
    (GitHub Flow, issues, pull requests, reviews).
  \item How to design software systems with clear architecture and
    separation of concerns (\texttt{lib} $\rightarrow$ \texttt{cli}
    $\rightarrow$ \texttt{bot}).
  \item How to write meaningful tests and use testing as a foundation
    for code quality and refactoring.
  \item How to build and maintain \texttt{CI} and \texttt{CD}
    pipelines for testing, releases, and deployments.
  \item How to develop and publish libraries, CLI tools, and services
    ready for real users.
  \item How to document projects properly (\texttt{README}, wiki,
    licenses) to support contributors and users.
  \item How to work with secrets, automation, and deployment in
    production-like environments.
  \item How to present, pitch, and publicly share an Open Source project.
\end{itemize}

\hlsection{Schedule}
\renewcommand{\arraystretch}{1.3}
\begin{tabular}{ll}
  \texttt{Week 1:} & Course overview, Open Source fundamentals, project idea
  and GitHub workflow \\
  \texttt{Week 2:} & Library-first design, testing mindset, and Test Driven
  Development in Python \\
  \texttt{Week 3:} & CI/CD pipelines, releases, licensing, and introduction to
  CLI tools \\
  \texttt{Week 4:} & CLI development, documentation, licensing decisions, and
  repository structure \\
  \texttt{Week 5:} & Writing effective READMEs and improving project
  documentation \\
  \texttt{Week 6:} & Deployment fundamentals, Docker basics, and
  production automation \\
  \texttt{Week 7:} & Pitch preparation, slide decks, and project presentation \\
  \texttt{Week 8:} & Final presentations, project retrospective, and
  future directions \\
\end{tabular}
\renewcommand{\arraystretch}{1.0}

\end{document}
