\documentclass[11pt]{article}

\usepackage[T2A]{fontenc}        % кириллица
\usepackage[utf8]{inputenc}      % UTF-8
\usepackage[russian,english]{babel} % локализация
\usepackage{lmodern}             % Latin Modern (CMR расширенный,
\usepackage{cmap}
% включает кириллицу)
\renewcommand{\rmdefault}{cmr}   % основной шрифт
\renewcommand{\ttdefault}{cmtt}  % моноширинный

\usepackage{textcomp}

\usepackage[left=3.5cm, right=3.5cm, top=2cm, bottom=2cm]{geometry}
\usepackage{lmodern}
\usepackage{microtype}
\usepackage[htt]{hyphenat}
\usepackage{enumitem}
\usepackage{graphicx}

\usepackage{xcolor}
\usepackage{titlesec}

\definecolor{accent}{HTML}{4C8047}

\newcommand{\hlsection}[1]{%
  \par\addvspace{10pt}%
  \noindent
  \colorbox{accent}{%
    \ttfamily\bfseries\color{white}%
    \strut#1%
  }%
  \par\addvspace{6pt}%
}

\usepackage{setspace}
\setlength{\parindent}{0pt}
\setlength{\parskip}{1pt}
\setstretch{0.95}

\usepackage{titlesec}
\titlespacing*{\section}{0pt}{1em}{0.5em}

\usepackage{xcolor}
\usepackage[normalem]{ulem}
\usepackage{hyperref}

\definecolor{linkgray}{RGB}{50,50,50}

\newcommand{\myhref}[2]{%
  \href{#1}{\textcolor{linkgray}{\underline{#2}}}%
}

\usepackage[htt]{hyphenat}       % переносы в typewriter (\texttt)
\hyphenation{до-ку-мен-ти-ро-ван-ных}
\hyphenation{го-то-вы-е}
\hyphenation{под-держ-ки}
\hyphenation{под-держ-ки}
\hyphenation{про-цесс}

\begin{document}

\section*{\texttt{Разработка Open Source проектов (EOSP)}}

\hspace{1cm}

\texttt{Серия лекций и семинаров от Антона Гришина\\
  Для старшеклассников из
  \myhref{https://xn--l1afu.xn--p1ai/}{ЦПМ} на кампусе
  \myhref{https://cu.ru}{Центрального университета}.\\
  Все материалы курса доступны в репозитории
  \myhref{https://github.com/alchemmist/eosp}{alchemmist/eosp} на GitHub.
}

\hlsection{Аннотация}
\textit{Проект создаёт систему для оценки вклада разработчиков на основе
  активности в GitHub, реализованную как Python-библиотека, CLI и
  Telegram-бот. Студенты изучают модульное проектирование, тестирование,
  CI/CD, документацию и реальные практики Open Source, с акцентом на
удобство использования и чистую, расширяемую архитектуру.}

\hlsection{Цель проекта}
Цель проекта — создать практическую систему для оценки вклада
разработчиков или команд на основе активности в GitHub. Система
включает Python-библиотеку для расчёта метрик, CLI для взаимодействия
с библиотекой и Telegram-бота для удобного доступа.

\hlsection{Преподаватель}
\begin{minipage}{0.75\textwidth}
  Антон — практикующий разработчик и преподаватель Python с более чем
  четырьмя годами опыта, участвующий в десятках репозиториев.
  Финалист Чемпионата России по спортивному программированию и
  победитель конференции \guillemotleft{Наука для жизни}\guillemotright.
  Автор блога:
  \myhref{https://alchemmist.xyz}{\footnotesize\texttt{alchemmist.xyz}}.
  Подробнее в \myhref{https://alchemmist.xyz/cv/ru/short}{резюме}.
\end{minipage}
\hspace{0.03\textwidth}
\begin{minipage}{0.2\textwidth}
  \includegraphics[width=6em]{anton.jpg}
\end{minipage}

\hlsection{Почему этот курс?}
HR-процессы и управление IT становятся всё сложнее, и оценивать
разработчиков по продуктивности, качеству кода, навыкам и вкладу
без чётких метрик крайне сложно. Этот курс обучает студентов
реальным практикам Open Source, предоставляя практический опыт
в проектировании модульных, поддерживаемых и хорошо документированных систем.

\hlsection{Методология}
Курс ориентирован на практику. С самой первой лекции студенты
получают конкретное задание: спроектировать и создать реальный проект.
Все последующие лекции и материалы строятся таким образом, чтобы
постепенно добавлять знания и инструменты для завершения и
улучшения проекта.

\hlsection{Результаты обучения}
К концу курса студенты будут знать:
\begin{itemize}[topsep=0pt, partopsep=0pt, itemsep=2pt, parsep=0pt]
  \item Как на практике проходит разработка в Open Source
    (GitHub Flow, issue, pull request, code review).
  \item Как проектировать архитектуру программ с чётким разделением
    обязанностей (\texttt{lib} $\rightarrow$ \texttt{cli}
    $\rightarrow$ \texttt{bot}).
  \item Как писать полезные тесты и использовать тестирование как
    основу качества кода и рефакторинга.
  \item Как строить и поддерживать \texttt{CI} и \texttt{CD} для
    тестирования, релизов и деплоя.
  \item Как создавать и публиковать библиотеки, CLI-инструменты и
    сервисы, готовые к использованию.
  \item Как правильно документировать проекты (\texttt{README}, wiki,
    лицензии) для поддержки участников и пользователей.
  \item Как работать с секретами, автоматизацией и деплоем в
    средах, приближённых к производственным.
  \item Как презентовать, защищать и публично демонстрировать
    Open Source-проект.
\end{itemize}

\hlsection{План курса}
\renewcommand{\arraystretch}{1.3}
\begin{tabular}{l p{12cm}}
  \texttt{Неделя 1:} & Обзор курса, основы Open Source, идея проекта
  и рабочий процесс GitHub \\
  \texttt{Неделя 2:} & Проектирование библиотеки, тестирование и TDD
  на Python \\
  \texttt{Неделя 3:} & \texttt{CI/CD}, релизы, лицензии и введение в
  CLI-инструменты \\
  \texttt{Неделя 4:} & Разработка CLI, документация, лицензии и
  структура репозитория \\
  \texttt{Неделя 5:} & Написание эффективных README и улучшение
  документации проекта \\
  \texttt{Неделя 6:} & Основы деплоя, базовый Docker и автоматизация
  процессов \\
  \texttt{Неделя 7:} & Подготовка презентаций, слайдов и демонстрация проекта \\
  \texttt{Неделя 8:} & Финальные презентации, ретроспектива проекта и
  планы на будущее \\
\end{tabular}
\renewcommand{\arraystretch}{1.0}

\end{document}
